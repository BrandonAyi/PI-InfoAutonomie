\section{Sensor}

L'API permet de recevoir les nouveaux capteurs détectés par le Raspberry. Ainsi, la plateforme peut connaître la liste des capteurs et associer une description et une localisation du capteur.

\paragraph{POST /api/sensor} permet d'ajouter un nouveau capteur à la plateforme web. Si le capteur est déjà présent, l'action est ignorée.
\begin{lstlisting}[language=json,emph={token, anAddress, aType},emphstyle={\bfseries\itshape},basicstyle={\ttfamily},caption=POST /api/sensor]
HEADER : {"AUTH":"token"}
BODY : {"address":"anAddress","type":"aType"}

# Exemple

HEADER : {"AUTH":"hj1456bsdg1bfsg846bg1sb125gfd"}
BODY : {"address":"153.111","type":"LIGHT"}
\end{lstlisting}

\texttt{type} peut prendre différentes valeurs parmi :
\begin{itemize}
\item LIGHT
\item TEMP
\item HUMIDITY
\item PRESENCE
\item DOOR
\item POWER
\end{itemize}


\section{Data}

\paragraph{POST /api/data} permet d'ajouter une nouvelle donnée à la plateforme web. Si la valeur est déjà présente, l'action est ignorée.
\begin{lstlisting}[language=json,emph={token, aTimestamp, aLabel,aValue,aMoteAddress,aType},emphstyle={\bfseries\itshape},basicstyle={\ttfamily},caption=POST /api/data]
HEADER : {"AUTH":"token"}
BODY : {"timestamp":"aTimestamp","label":"aLabel",
									"value":"aValue","mote":"aMoteAddress.aType"}

# Exemple
HEADER : {"AUTH":"hj1456bsdg1bfsg846bg1sb125gfd"}
BODY : {"timestamp":1411848808,"label":"temperature","value":24.0,"mote":"219.98.TEMP"}
\end{lstlisting}

Type peut prendre les même valeurs que précédemment.